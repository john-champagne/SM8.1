\documentclass[11pt]{article}

\title{\textbf{SM8.1 Specifications}\\Simple Microprocessor 8-Bit\\Version 1.0}
\author{John Champagne}
\date{\today}
\begin{document}

\maketitle
\section{Overview}
The SM8.1 is a microprocessor specification designed for my fall 2017 honors project. It is designed for simplicity rather than robustness or power, and should serve as a proof of concept.
\subsection{Registers}
The SM8.1 has 5 8-Bit registers, labeled $A (00)$, $B(01)$, $C(10)$, and $D(11)$.\\
$A$ serves as the primary register, while B,C, and D are general purpose registers.
\subsection{RAM}
The address space of the memory is 8 bits wide, which yields a full 256 memory locations, each 8 bits wide. This gives a total of 256 bytes of RAM.
\subsection{ROM}
The ROM is similar to the RAM. 8 bits of address space and 256 bytes of ROM, each representing an instruction. There is an 8 bit register called the instruction pointer ($IP$) which points to the current instruction.
\newpage
\section{Instruction Set}
\subsection{Overview}
The 8 bit instructions are organized like so.

$$\big[7\ 6\ 5\ 4\ 3\ 2\ 1\ 0\big]$$

Bit 7 differentiates between a load instruction and a computation instruction. Bits 6-2 represent OP codes. Bits 1-0 represent a register to pass as an argument.
\subsection{Instruction Table}
\begin{tabular}{l | l | l}
Symbol & Meaning & Example\\
\hline
$A$,$B$,$C$,$D$ & A specific register.\\
$A_u$ & The unsigned value of register A & \\
$R_x$ & A general register, specified by bits 1-0. & When $x=01$, $R_x = B$\\
$[R_x]$ & The memory address pointed to by the register $R_x$ &\\
$\rightarrow$ & Stores the value on the left into the value on the right. & $\mathrm{0xff} \rightarrow A$\\
$+$ & Arithmetic addition. & $1 + 3 = 4$\\
$-$ & Arithmetic subtraction. & $3 - 2 = 1$\\
$|$ & Logical OR. & $010 | 100 = 110$\\
$\&$ & Logical AND. & $101 \& 011 = 001$\\
$\bigotimes$ & Logical XOR. & $110 \bigotimes 100 = 010$\\
$>>$ & Logical right shift.& $110 >> 2 = 001$\\
$<<$ & Logical left shift. & $101 << 1 = 010$\\
$\mathrm{jmp}(R_x)$ & Change the instruction pointer to the value in $R_x$&\\
$?$ & Conditional. Only do the following if the former is true.& $3 < 5 ? \mathrm{jmp}(R_x)$\\

\end{tabular}

\begin{tabular}{|l | l | l |}
\hline
\multicolumn{3}{|c|}{ \textbf{Instruction Set}} \\
\hline
\textbf{bin} & \textbf{inst} & \textbf{asm}\\
\hline 
$1\,xxxxxxx$ & $x \rightarrow A$ & ld x;  \\
\hline
$000\,xxx\,x_x x_x$&\multicolumn{2}{c|}{ \textbf{Arithmetic Instructions}} \\
\hline
$000\,000\,x_x x_x$ & $A + R_x \rightarrow A$ & add A, X; \\
$000\,001\,x_x x_x$ & $A - R_x \rightarrow A$ & sub A, X; \\
$000\,010\,x_x x_x$ & $A | R_x \rightarrow A$ & or A, X; \\
$000\,011\, x_x x_x$ & $A \& R_x \rightarrow A$ & and A, X; \\
$000\,100\, x_x x_x$ & $A \bigotimes R_x \rightarrow A$ & xor A, X; \\
$000\,101\, x_x x_x$ & $A >> R_x \rightarrow A$ & rsh A, X; \\
$000\,110\, x_x x_x$ & $A << R_x \rightarrow A$ & lsh A, X; \\
\hline
$001\,xxx\,x_x x_x$&\multicolumn{2}{c|}{ \textbf{Memory Instructions}} \\
\hline
$001\,000\, x_x x_x$ & $[R_x] \rightarrow A$ & rd X; \\
$001\,001\, x_x x_x$ & $A \rightarrow [R_x]$ & wr X; \\
$001\,010\, x_x x_x$ & $R_x \rightarrow A$ & mov X, A; \\
$001\,011\, x_x x_x$ & $A \rightarrow R_x$ & mov A, X; \\
\hline
$010\,xxx\,x_x x_x$&\multicolumn{2}{c|}{ \textbf{Conditional Instructions}} \\
\hline
$010\,000\, x_x x_x$ & $A > 0 \,?\, \mathrm{jmp}(R_x)$ & jg X; \\
$010\,001\, x_x x_x$ & $A \geq 0 \,?\, \mathrm{jmp}(R_x)$ & jge X; \\
$010\,010\, x_x x_x$ & $A = 0 \,?\, \mathrm{jmp}(R_x)$ & je X; \\
$010\,011\, x_x x_x$ & $A < 0 \,?\, \mathrm{jmp}(R_x)$ & jl X; \\
$010\,100\, x_x x_x$ & $A \leq 0 \,?\, \mathrm{jmp}(R_x)$ & jle X; \\
$010\,101\, x_x x_x$ & $A_u > 0 \,?\, \mathrm{jmp}(R_x)$ & jgu X; \\
$010\,110\, x_x x_x$ & $A_u \geq 0 \,?\, \mathrm{jmp}(R_x)$ & jgeu X; \\
$010\,111\, x_x x_x$ & $\mathrm{jmp}(R_x)$ & jmp X; \\
\hline
\end{tabular}


\end{document}
